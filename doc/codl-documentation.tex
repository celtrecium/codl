\documentclass{article}

\author{Maxim ``Celtrecium''}
\title{The CODL library documentation}

\usepackage[a4paper, left=20mm, right=20mm, top=20mm, bottom=20mm]{geometry}
\usepackage{listings}
\usepackage{color}

\definecolor{gray}{rgb}{0.5,0.5,0.5}
\definecolor{ccomment}{rgb}{0.56, 0.56, 0.55}
\definecolor{ckeyword}{rgb}{0.26, 0.44, 0.68}
\definecolor{cstring}{rgb}{0.24, 0.60, 0.62}

\lstset{
  language=C,
  basicstyle=\small\tt,
  numbers=left,
  numberstyle=\small\tt\color{gray},
  stepnumber=1,
  numbersep=5pt,
  backgroundcolor=\color{white},
  showspaces=false,
  showstringspaces=false,
  showtabs=false,
  % frame=single,
  tabsize=2,
  captionpos=t,
  breaklines=true,
  breakatwhitespace=false,
  escapeinside={\%*}{*)},
  commentstyle=\color{ccomment},
  keywordstyle=\color{ckeyword},
  stringstyle=\color{cstring}
}

\newcommand{\fstep}{\vspace{3mm}\noindent}

\begin{document}

{\vspace{10cm}\bf\Huge\maketitle}

\newpage

\tableofcontents

\newpage

\section{Introduction}

The CODL library is a library that allows you to create applications using
console graphics. It was inspired by ncurses-based applications.

At the moment this library is very simple and contains basic functions. In the
future, it is planned to create an add-on with the ability to create an user
interface.

\section{Basic functions and concepts}
The main role in the library is played ``windows'' --- these are
buffers that have a certain position on the screen, width, height, layer,
and such internal settings as the background color, symbol color, cursor
position in the buffer, alpha channel, and window visibility.

\subsection{Initialize, end program}

\subsubsection{codl\_initialize}
{\tt int codl\_initialize(void)}

\fstep{} Required to start working with the library.

\noindent\smallskip Initializes two screen buffers and a terminal window, clears the contents
of the terminal emulator.

\subsubsection{codl\_end}
{\tt int codl\_end(void)}

\fstep{} Required to end working with the library.

\noindent\smallskip Clears all memory allocated for window buffers.

{\vspace{5mm}\noindent\bf\large For example:}

\begin{lstlisting}[language=C]
#include <codl.h>

int
main (void)
{
  codl_initialize ();
  ...
  codl_end ();
  
  return EXIT_SUCCESS;
}
\end{lstlisting}

\subsection{Window type}
{\tt codl\_window} is a structure that includes the window buffer and its parameters:

\subsubsection*{struct codl\_window *parent\_win}
Pointer on the parent window (By default, it refers to {\tt term\_window})

\subsubsection*{int x\_position, int y\_position}
Absolute location of the window on the X, Y axis

\subsubsection*{int ref\_x\_position, int ref\_y\_position}
Location of the window on the X, Y axis, relative to the parent window

\subsubsection*{int width, int height}
Size of the window

\subsubsection*{int layer}
Window layer. For example: The window on layer 1 will be below the window on
layer 2.

\subsubsection*{int cursor\_pos\_x, int cursor\_pos\_y}
Buffer cursor location on the X, Y axis

\subsubsection*{int s\_cur\_pos\_x, int s\_cur\_pos\_y}
Saved cursor buffer location (needed for save and restore cursor position)

\subsubsection*{int colour\_bg, int colour\_fg}
Buffer background and foreground colors

\subsubsection*{char alpha}
Alpha channel setting ({\tt CODL\_ENABLE} or {\tt CODL\_DISABLE})
If this attribute is activated, the empty space of the window will
not be displayed on the overall composition.

\subsubsection*{char text\_attribute}
Text attribute setting ({\tt CODL\_BOLD, CODL\_ITALIC, CODL\_UNDERLINE,
CODL\_CROSSED\_OUT, CODL\_DIM})

\subsubsection*{int window\_visible}
Window visibility setting ({\tt CODL\_ENABLE} or {\tt CODL\_DISABLE})

\subsubsection*{char ***window\_buffer}
Window buffer

\subsection{Create and destroy window}

\subsubsection{codl\_create\_window}
{\tt codl\_window *codl\_create\_window(codl\_window *p\_win, int layer, int x\_pos, int y\_pos, int width, int height)}

\medskip Parameters:
\begin{itemize}
\item{{\tt *p\_win} --- parent window pointer (If the pointer is {\tt NULL}, the parent window is set as term\_window)}
\item{{\tt layer} --- layer of the window being created}
\item{{\tt x\_pos, y\_pos} --- location of the window on the X, Y axis, relative to the parent window}
\item{{\tt width, heigth} --- size of the window being created}
\end{itemize}

\subsubsection{codl\_destroy\_window}
{\tt int codl\_destroy\_window(codl\_window *win)}

\fstep{} This function has only one argument --- a pointer to the window to destroy

\newpage

{\vspace{5mm}\noindent\bf\large For example:}

\begin{lstlisting}[language=C]
#include <codl.h>

int
main (void)
{ 
  /* We can't create a window until the library is initialized */
  codl_window *first_win = NULL; 
  codl_window *second_window = NULL;

  codl_initialize ();
  
  /* We create the first window with next parameters:
   * Parent window               = NULL (Refers to the term_window) 
   * Layer of the window         = 1
   * Rel. position on the X axis = 5
   * Rel. position on the Y axis = 5
   * Width                       = 20
   * Height                      = 10 
   */
  first_window = codl_create_window (NULL, 1, 5, 5, 20, 10);
  
  /* We are also creating a second window. Now its parent 
     window will be our first window
   */
  second_window = codl_create_window (first_window, 2, 2, 10, 5);

  /*
   *                      Some code
   */

  /* If you need to get rid of one of the windows,
   * you can call the code_destroy_window function.
   * 
   * For example:
   */

  codl_destroy_window (second_window);
  
  /* Now our second window has been deleted and the pointer to it is NULL */

  /*
   *                    Some more code
   */

  codl_end ();
  
  return EXIT_SUCCESS;
}
\end{lstlisting}

\newpage

\subsection{Image type}
The {\tt cool\_image} type partially repeats the code\_window type, except that
it has only two attributes: width and height

\begin{itemize}
\item{\tt int width}
\item{\tt int height}
\item{\tt char ***image\_buffer}
\end{itemize}

This buffer is needed to repeatedly load the image from it, so as not to read
the image directly from the disk each time.

\subsubsection{codl\_image\_to\_window}
{\tt int codl\_image\_to\_window(codl\_window *win, codl\_image *img,
                int x\_pos, int y\_pos, int x\_reg, int y\_reg, int width, int height)}

\fstep{} This function transfers the area of image (which is
selected by the parameters x\_reg, y\_reg, width, height) from codl\_image to the
window buffer at coordinates X, Y.

\subsubsection{codl\_save\_buffer\_to\_file}
{\tt int codl\_save\_buffer\_to\_file(codl\_window *win, const char *filename)}

\fstep{} In the first argument, we specify the window whose buffer we want to save.
In the second argument, we specify the name of the file in which the buffer will be saved.

\subsubsection{codl\_load\_buffer\_from\_file}
{\tt int codl\_load\_buffer\_from\_file(codl\_window *win, const char *filename,
                int x\_pos, int y\_pos)}

\fstep{} This function loads an image from a file directly into
the window buffer at X, Y coordinates.

\subsubsection{codl\_load\_image}
{\tt codl\_image *codl\_load\_image(const char *filename)}

\fstep{} This function loads an image from a file into the
codl\_image buffer

\section{Library error system}
The library error system is implemented quite simply: if a library function
fails, it returns a null value and uses the {\tt codl\_set\_fault} function to
determine the cause of this error.

You can get the {\tt CODL\_FAULTS} enum value of this error using
{\tt code\_get\_fault\_enum} or get a string explaining the error using
{\tt codl\_get\_fault\_string}.

\section{Setter functions}
This section lists all the functions for manipulating windows with a brief
description of them.

\subsection{Color setters}

The first argument in these functions is the window to which the property is applied.

\subsubsection{codl\_set\_colour}
{\tt int codl\_set\_colour(codl\_window *win, int bg, int fg)}

\fstep{} This function takes three parameters:
\begin{itemize}
\item{A pointer to a window}
\item{Background color (0 to 256)}
\item{Foreground color (0 to 256)}
\end{itemize}

\subsection{Text attribute setters}
The following attributes are available for attribute setters:
\begin{enumerate}
\item{{\tt CODL\_NO\_ATTRIBUTES} --- zero attribute}
\item{{\tt CODL\_BOLD} --- makes the text bold}
\item{{\tt CODL\_ITALIC} --- makes the text italicized}
\item{{\tt CODL\_UNDERLINE} --- makes the text underlined}
\item{{\tt CODL\_CROSSED\_OUT} --- makes the text crossed out}
\item{{\tt CODL\_DIM} --- makes the text dim}  
\end{enumerate}

\noindent\vspace{3mm} You can also combine these attributes by using a logical OR ({\tt ||})

\noindent\vspace{3mm} For example:

{\tt codl\_set\_attribute (window\_name, CODL\_BOLD || CODL\_ITALIC || CODL\_UNDERLINE);}

\subsubsection{codl\_set\_attribute}
{\tt int codl\_set\_attribute(codl\_window *win, char attribute)}

\fstep{} This function sets the window text attributes completely.

\subsubsection{codl\_add\_attribute}
{\tt int codl\_add\_attribute(codl\_window *win, char attribute)}

\noindent\vspace{3mm} This function adds a text attributes to the already set
ones.

\subsubsection{codl\_remove\_attribute} 
{\tt int codl\_remove\_attribute(codl\_window *win, char attribute)}

\fstep{} This function deletes the attributes specified in the
argument.

\subsection{Window attribute setters}

\subsubsection{codl\_set\_alpha}
{\tt int codl\_set\_alpha(codl\_window *win, CODL\_SWITCH alpha);}

\fstep{} This function enables or disables the alpha mode of the window.

The first argument in this function is the window to which the property is applied.
The second argument can take two values: {\tt CODL\_ENABLE} or {\tt CODL\_DISABLE}

\subsubsection{codl\_set\_window\_visibility}
{\tt int codl\_set\_window\_visible(codl\_window *win, CODL\_SWITCH visible)}

\fstep{} This function enables or disables window visibility.

The second argument can take two values: {\tt CODL\_ENABLE} or {\tt CODL\_DISABLE}

\subsubsection{codl\_set\_cursor\_position}
{\tt int codl\_set\_cursor\_position(codl\_window *win, int x\_pos, int y\_pos)}

\fstep{} This function sets the position of the cursor in the buffer by X, Y coordinates.
If the horizontal position overflows, the buffer is shifted down.
If the vertical position overflows, the cursor moves to the next line.

\subsubsection{codl\_save\_cursor\_position}
{\tt int codl\_save\_cursor\_position(codl\_window *win)}

\fstep{} This function saves cursor position in the window to a special field
it the {\tt codl\_window} structure.

\subsubsection{codl\_restore\_cursor\_position}
{\tt int codl\_restore\_cursor\_position(codl\_window *win)}

\fstep{} This function restores cursor position from {\tt s\_cur\_pos\_}* fields
of the {\tt codl\_window} structure.

\subsubsection{codl\_resize\_window}
{\tt int codl\_resize\_window(codl\_window *win, int width, int height)}

\fstep{} This function sets the size of the window (width, length).

\subsubsection{codl\_set\_window\_position}
{\tt int codl\_set\_window\_position(codl\_window *win, int new\_x\_pos, int new\_y\_pos)}

\fstep{} This function sets the position of the window in X, Y
coordinates relative to its parent window.

\subsubsection{codl\_set\_layer}
{\tt int codl\_set\_layer(codl\_window *win, int layer)}

\fstep{} This function sets the layer on which the window

\subsubsection{codl\_window\_clear}
{\tt int codl\_window\_clear(codl\_window *win)}

\fstep{} This function clears the window buffer.

\subsection{Terminal attribute setters}
It is not recommended to use these functions while working with the library
(except for codl\_clear(if this function is followed by codl\_redraw or the
program terminates) and codl\_monochrome\_mode)

\subsubsection{codl\_cursor\_mode}
{\tt void codl\_cursor\_mode(CODL\_CURSOR cur)}

\fstep{} This function sets the terminal cursor mode: {\tt CODL\_SHOW} or
{\tt CODL\_HIDE}

\subsubsection{codl\_echo}
{\tt int codl\_echo(void)}

\fstep{} Enables echo mode (when keyboard input is displayed on stdout).

\subsubsection{codl\_noecho}
{\tt int codl\_noecho(void)}

\fstep{} Disables echo mode.

\subsubsection{codl\_monochrome\_mode}
{\tt void codl\_monochrome\_mode(CODL\_SWITCH mode)}

\fstep{} Enables monochrome mode (text does not have colors and attributes set).

\subsubsection{codl\_clear}
{\tt void codl\_clear(void)}

\fstep{} Clears the terminal screen (not terminal window)

\subsection{Primitive setters (Frame setters)}
Frame setters work like this: they set parameters for drawing a frame like this

\begin{verbatim}

 4 ----- 2 ----- 5
 |               |
 |               |
 0               1
 |               |
 |               |
 6 ----- 3 ----- 7

\end{verbatim}

In function arguments, these parts of the frame can be denoted by a prefix
({\tt fg\_} or {\tt ch\_}) and the number of this part. 

\subsubsection{codl\_set\_frame\_colours}
{\tt int codl\_set\_frame\_colours(int fg\_0, int fg\_1, int fg\_2, int fg\_3,
                int fg\_4, int fg\_5, int fg\_6, int fg\_7)}
              
\fstep{} Sets the colors for drawing the frame (to understand the
arguments, follow the instructions above)

\subsubsection{codl\_set\_frame\_symbols}
{\tt int  codl\_set\_frame\_symbols(char *ch\_0, char *ch\_1, char *ch\_2,
                char *ch\_3, char *ch\_4, char *ch\_5, char *ch\_6, char *ch\_7)}

\fstep{} Sets the characters that the frame will be drawn with.
Also in the library there are preset symbols for drawing a frame.

\subsection{Error system setters}
\subsubsection{codl\_set\_fault}
{\tt int codl\_set\_fault(CODL\_FAULTS fault\_en, const char *fault\_str)}

\fstep{} This function sets the error value to the internal library buffer.

The first argument is a CODL\_FAULTS enum value:
\begin{itemize}
\item{{\tt CODL\_MEMORY\_ALLOCATION\_FAULT} --- error occurs when allocating memory}
\item{{\tt CODL\_NULL\_POINTER} --- error occurs when pointer is {\tt NULL} value}
\item{{\tt CODL\_INVALID\_SIZE} --- error occurs when the size is not suitable}
\item{{\tt CODL\_NOT\_INITIALIZED} --- error occurs when the library is not initialized}
\end{itemize}

The second argument is a string with an explanation of the error.

\subsection{Tab width setter}
By default tab width equals 8 spaces

\subsubsection{codl\_set\_tab\_width}
{\tt void codl\_set\_tab\_width(int width)}

\fstep{} Sets a tab width.

\subsection{Image setters}
\subsubsection{codl\_clear\_image}
{\tt int codl\_clear\_image(codl\_image *img)}

\fstep{} Clears the {\tt codl\_image} buffer

\section{Getter functions}

\subsection{Window getters}

\subsubsection{codl\_get\_num\_of\_wins}
{\tt int codl\_get\_num\_of\_wins(void)}
              
\fstep{} This function returns the number of windows

\subsubsection{codl\_get\_term}
{\tt codl\_window *codl\_get\_term(void)}

\fstep{} This function returns a pointer to the term\_window window.
This window can be used for drawing, writing. You can also use it to find out
the size of the terminal screen.

\subsection{Terminal getters}

\subsubsection{codl\_get\_term\_size}
{\tt int codl\_get\_term\_size(int *width, int *height)}

\fstep{} This function takes as arguments pointers to variables
of the int type, in which the width and height of the terminal screen will be written.

If you want to know the size of the terminal screen, you'd better do it with
the code\_get\_term getter.

For example: {\tt codl\_get\_term ()->width}, {\tt codl\_get\_term ()->height}

\subsubsection{codl\_resize\_term}
{\tt int codl\_resize\_term(void)}

\fstep{} This function checks whether the size of the terminal has
changed, and if it has changed, sets the new size of the terminal window and
returns the value 1.

\subsection{Error system getters}

\subsubsection{codl\_get\_fault\_string}
{\tt char *codl\_get\_fault\_string(void)}

\fstep{} This function returns a pointer to a string with an error
explanation.

\subsubsection{codl\_get\_fault\_enum}
{\tt CODL\_FAULTS codl\_get\_fault\_enum(void)}

\fstep{} This function returns the CODL\_FAULTS enum value.

\subsection{Tab width getter}
\subsubsection{codl\_get\_tab\_width}

{\tt int codl\_get\_tab\_width(void)}

\fstep{} This function returns the value of the tab width.

\subsection{Input getters}

\subsubsection{codl\_get\_key}
{\tt unsigned int codl\_get\_key(void)}

\fstep{} If the key was pressed, this function returns:

\begin{itemize}
\item{The ASCII value of the key}
\item{The value of the key that is listed in the CODL\_KEY enum}
\item{The value of CODL\_KEY\_UNICODE, in the case of which we can call the
  getter codl\_get\_stored\_key to get the unicode value of the key}
\end{itemize}

Or 0 value if a key has not been pressed.

\subsubsection{codl\_get\_stored\_key}
{\tt char *codl\_get\_stored\_key(void)}

\fstep{} This function returns a pointer to the buffer where the
unicode key was written (the size of this buffer is 4, because the maximum
UTF-8 character size is 4 bytes)

You can also use the {\tt strcmp} function from the standard library to compare a
pressed key with a unicode character.

{\vspace{5mm}\noindent\bf\large For example:}

\begin{lstlisting}[language=C]
#include <codl.h>

int
main (void)
{
  unsigned int key = 0;

  codl_initialize ();

  /* The loop will end if the resulting key is equal to the Escape 
   * key code (this code can be viewed in CODL_KEY enum) 
   */
  while ((key = codl_get_key ()) != CODL_KEY_ESC)
    {
      switch (key)
        {
        case 0:
          continue;
        
        case CODL_KEY_UP:
          codl_write (codl_get_term (), "Oh, honey, you pushed the up button..." 
                                        " Push something else;)\n");

          break;

        case CODL_KEY_UNICODE:
          codl_write (codl_get_term (), "Wow, you hit the button ");
          codl_write (codl_get_term (), codl_get_stored_key ());
          codl_write (codl_get_term (), "\n");

          break;

        default:
          codl_write (codl_get_term (), "You don't spoil me... Can you press "
                                        "the up key or some non-ASCII key?\n");

          break;
        }
      
      codl_display ();
    }

  codl_end ();

  return EXIT_SUCCESS;
}
\end{lstlisting}

\subsection{String getters}
\subsubsection{codl\_strlen}
{\tt size\_t codl\_strlen(const char *string)}

\fstep{} This function is analog of {\tt strlen} function from
{\tt string.h}

\subsubsection{codl\_string\_length}
{\tt size\_t codl\_string\_length(const char *string)}

\fstep{} This function counts the number of characters in a string.
Supports UTF-8

\section{Functions for manipulating the buffer}
This section contains functions for working with the {\tt codl\_window} buffer.

\subsection{codl\_buffer\_scroll\_down}
{\tt int codl\_buffer\_scroll\_down(codl\_window *win, int down)}
           
\fstep{} This function shifts the contents of the window buffer
by a certain number of characters down.

\subsection{codl\_buffer\_scroll\_up}
{\tt int codl\_buffer\_scroll\_up(codl\_window *win, int down)}
           
\fstep{} This function shifts the contents of the window buffer
by a certain number of characters up.

\section{Functions for writing and drawing primitives}
The functions of this section output text to the terminal with the attributes
that you set with the color and text setters (except for the codl\_frame
function, which has its own setters
(the frame takes the background color from the window attributes))


\subsection{Functions for writing}
\subsubsection{codl\_write}
{\tt int codl\_write(codl\_window *win, char *string)}

\fstep{} This function writes a string to the window buffer.
It is the main function of writing to a window buffer. Supports parsing ANSI
sequences. For example:
\begin{lstlisting}[language=C]
codl_write (window_name, "\033[1mHello world!\033[0m");
\end{lstlisting}

\subsubsection{codl\_replace\_attributes}
{\tt int codl\_replace\_attributes(codl\_window *win, int x0\_pos, int y0\_pos,
                int x1\_pos, int y1\_pos)}

\fstep{} This function replaces the text attributes with those that you
previously set using color and text setters in the region marked with
coordinates $x_0, y_0, x_1, y_1$

\subsection{Functions for drawing primitives}
\subsubsection{codl\_line}
{\tt int codl\_line(codl\_window *win, int x1, int y1, int x2, int y2,
                char *symbol)}

\fstep{} This function draws a line at coordinates $x_0, y_0, x_1, y_1$ using a
character, which is specified as the last argument using a string literal

\subsubsection{codl\_rectangle}
{\tt int codl\_rectangle(codl\_window *win, int x0\_pos, int y0\_pos,
                int x1\_pos, int y1\_pos, char *symbol)}

\fstep{} This function draws a rectangle at coordinates $x_0, y_0, x_1, y_1$ using a
character, which is specified as the last argument using a string literal

\subsubsection{codl\_rectangle}
{\tt int codl\_rectangle(codl\_window *win, int x0\_pos, int y0\_pos,
                int x1\_pos, int y1\_pos, char *symbol)}

\fstep{} This function draws a frame at coordinates $x_0, y_0, x_1, y_1$ using a
characters set by {\tt codl\_set\_frame\_symbols} function and with colors set by
{\tt codl\_set\_frame\_colours} function.

\section{Functions for working with memory}
\subsection{Memory (re-)allocation functions}
This subsection contains wrappers over the memory allocation functions from
the standard library. These are safe functions that have integration with the
error system of this library.

\subsubsection{codl\_malloc\_check}
{\tt void *codl\_malloc\_check(size\_t size)}

\fstep{} This function allocates {\tt size} bytes on the heap and returns a pointer
to the beginning of this area.

\subsubsection{codl\_realloc\_check}
{\tt void *codl\_realloc\_check(void *ptrmem, size\_t size)}
              
\fstep{} This function reallocates memory blocks. The size of the memory
block referred to by the {\tt ptrmem} parameter is changed to size bytes.
The memory block can shrink or grow in size.

\subsubsection{codl\_calloc\_check}
{\tt void *codl\_calloc\_check(size\_t number, size\_t size)}

\fstep{} The calloc function allocates a block of memory for an array of {\tt number}
elements, each of which is {\tt size} bytes, and initializes all of its bits to
zeros. As a result, a memory block of {\tt number} * {\tt size} bytes is allocated, and
the entire block is filled with zeros.

\subsection{Set and copy memory functions}
This subsection contains the safe counterparts of the standard library functions.

\subsubsection{codl\_memset}
{\tt int codl\_memset(void *dest, codl\_rsize\_t destsize, int ch, codl\_rsize\_t count)}

\fstep{} This function fills the {\tt count} bytes of memory at {\tt dest} with {\tt ch}. If
{\tt count} is bigger than {\tt destsize}, the function sets {\tt destsize} bytes of memory.

\subsubsection{codl\_memcpy}
{\tt int codl\_memcpy(void *dest, codl\_rsize\_t destsize, const void *src,
  codl\_rsize\_t count)}

\fstep{} This function copies {\tt count} bytes of memory from {\tt src} to {\tt dest}. If {\tt count}
is greater than {\tt destsize}, the function copies {\tt destsize} bytes of memory. This
function is protected from memory overlap.

\section{Display functions}
\subsection{codl\_display}
{\tt int codl\_display(void)}

\fstep{} This function is engaged in displaying the picture and all its changes
on the screen of your terminal. This is the main display function, in most cases
you need to use it.

\subsection{codl\_redraw}
{\tt int codl\_redraw(void)}

\fstep{} This function completely redraws the image on the screen.
It may be useful after using the {\tt codl\_clear} function, in other cases
it is better to refrain from using it.

\subsection{codl\_redraw\_diff}
{\tt int codl\_redraw\_diff(void)}

\fstep{} This function can re-display the changes that have occurred on the
screen. The function is needed only in theory, in practice it has not yet
been used.

\section{Other functions}
This section contains features that do not fall into other categories

\subsection{codl\_itoa}
{\tt char *codl\_itoa(int num, char *string)}

\fstep{} This function converts an int value to a string.

\subsection{codl\_input\_form}
{\tt int codl\_input\_form(codl\_window *win, char **str, int pos\_x, int pos\_y, size\_t size)}

\fstep{} This function creates a form for input in a {\tt win} window of size {\tt size}
characters, which will be located at the x, y coordinates relative to the specified
window. Also, this function accepts a pointer to a string in order to write the
result into it after the end of the work. The memory for the row is allocated on
the heap, so remember to clear the memory after you finish.

\newpage

\section{Some more examples}

\subsection{Image demo}

\begin{lstlisting}[language=C]
#include <codl.h>

int
main (void)
{
  codl_window *win   = NULL;
  codl_window *s_win = NULL;
  codl_image  *img   = NULL;
  
  codl_initialize ();
  /* Creating window */
  win = codl_create_window (NULL, 1, 5, 5, 24, 8);

  /* Creating centered child window of "win" */
  s_win = codl_create_window (win, 2, 2, 2, win->width - 4, win->height - 4);
  
  /* Setting window color attributes for drawing rectangle for fill window 
   * buffer with solid color 
   */
  codl_set_colour (win, CODL_BRIGHT_GREEN, CODL_DEFAULT_COLOUR);

  /* Fill the window buffer with rectangle */
  codl_rectangle (win, 0, 0, win->width, win->height, " ");

  /* Draw window frame with default frame settings and bright green 
   * background color 
   */
  codl_frame (win, 0, 0, win->width, win->height);

  /* Setting cursor position, text and colour attributes for writing */
  codl_set_cursor_position (win, 6, 1);
  codl_set_colour (win, CODL_BLUE, CODL_BRIGHT_WHITE);
  codl_set_attribute (win, CODL_BOLD | CODL_UNDERLINE);

  /* Write "Hello world!" */
  codl_write (win, "Hello world!");

  /* Write some text to the second window with default attributes */
  codl_write (s_win, "This is some text in second window :P\nYou wrote: ");

  /* Save the window buffer of "win" to file "file_image.cdl" */
  codl_save_buffer_to_file (win, "file_image.cdl");

  /* Load our file to image buffer "img" */
  img = codl_load_image ("file_image.cdl");

  /* Load image from image buffer to terminal window buffer */
  codl_image_to_window(codl_get_term (), img, 14, 15, 0, 0,
                       img->width, img->height);

  /* Free our pointer after using */
  codl_clear_image (img);

  /* Display our results */
  codl_display ();

  /* As a result, we have a copy of the image from the win window
   * in our terminal window.
   */
  
  codl_end ();
  
  return EXIT_SUCCESS;
}
\end{lstlisting}

\subsection{Small demo}

\begin{lstlisting}[language=C]
#include <codl.h>

int
main (void)
{
  codl_window *win   = NULL;
  codl_window *s_win = NULL;
  char *str          = NULL;
  unsigned int key   = 0;
  
  codl_initialize ();
  /* Creating window */
  win   = codl_create_window (NULL, 1, 5, 5, 24, 8);

  /* Creating centered child window of "win" */
  s_win = codl_create_window (win, 2, 2, 2, win->width - 4, win->height - 4);
  
  /* Setting window color attributes for drawing rectangle for fill window 
   * buffer with solid color 
   */
  codl_set_colour (win, CODL_BRIGHT_GREEN, CODL_DEFAULT_COLOUR);

  /* Fill the window buffer with rectangle */
  codl_rectangle (win, 0, 0, win->width, win->height, " ");

  /* Draw window frame with default frame settings and bright green 
   * background color 
   */
  codl_frame (win, 0, 0, win->width, win->height);

  /* Setting cursor position, text and colour attributes for writing */
  codl_set_cursor_position (win, 6, 1);
  codl_set_colour (win, CODL_BLUE, CODL_BRIGHT_WHITE);
  codl_set_attribute (win, CODL_BOLD | CODL_UNDERLINE);

  /* Write "Hello world!" */
  codl_write (win, "Hello world!");

  /* Write some text to the second window with default attributes */
  codl_write (s_win, "This is some text in second window :P\nYou wrote: ");

  /* Prompt the user to enter a string */
  codl_set_colour (win, CODL_CYAN, CODL_BRIGHT_WHITE);
  codl_input_form(win, &str, 6, win->height - 2, 11);

  codl_set_attribute (s_win, CODL_BOLD);
  codl_write (s_win, str);

  /* Free memory after using codl_input_form function */
  free (str);
  
  /* Display our results */
  codl_display ();

  /* Create a loop in which the user can move the main window using the arrows
   */
  while ((key = codl_get_key ()) != CODL_KEY_ESC)
    {
      switch (key) {
      case 0:
        continue;
        
      case CODL_KEY_RIGHT:
        codl_set_window_position (win, win->x_position + 1, win->y_position);
        break;
        
      case CODL_KEY_LEFT:
        codl_set_window_position (win, win->x_position - 1, win->y_position);
        break;

      case CODL_KEY_UP:
        codl_set_window_position (win, win->x_position, win->y_position - 1);
        break;

      case CODL_KEY_DOWN:
        codl_set_window_position (win, win->x_position, win->y_position + 1);
        break;
      }

      codl_display ();
    }
  
  codl_end ();
  
  return EXIT_SUCCESS;
}
\end{lstlisting}

\end{document}
